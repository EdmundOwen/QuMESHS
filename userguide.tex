\documentclass[12pt]{article}
\usepackage[english]{babel}
\usepackage{amsmath,amsthm}
\usepackage{amsfonts}
\usepackage{amssymb}
\usepackage{bbold}              % for unitary matrix symbol
\usepackage{braket}
\usepackage{graphicx}
\usepackage{color}
\usepackage{fixltx2e}


\newcommand{\red}[1]{{\color{red} \it #1}}
\newcommand{\dd}{\text{d}}
\newcommand{\identity}{\mathbb{I}}
\newcommand{\overlap}[2]{\left\langle #1 | #2 \right\rangle}
\newcommand{\overlapint}[3]{\left\langle #1 \right| #2 \left| #3 \right\rangle}
\newcommand{\expectation}[1]{\left\langle #1 \right\rangle}
\newcommand{\modulus}[1]{\left| #1 \right|}
\newcommand{\intinfty}{\int_{- \infty}^{\infty}}
\newcommand{\dealii}{\texttt{deal.II}}

\newcommand{\commentout}[1]{}


\begin{document}

\title{QuMESHS User Guide}%
\author{Edmund T. Owen}%
\date{\today}%
% ----------------------------------------------------------------
\begin{abstract}
  QuMESHS is designed to self-consistently solve the charge-potential equations
  for layered semiconductor devices.  The electrostatic potential is calculated using
  finite-element methods and the density can be calculated using a variety of
  approximations in 1-, 2- and 3-D.  Convergence is achieved using a modified
  Newton-type iteration developed by Bank and Rose which allows for accelerated
  convergence.
\end{abstract}
\maketitle
% ----------------------------------------------------------------

\tableofcontents
\newpage


\section{Overview}

QuMESHS is a numerical code for solving finding the electron densities in quantum mesoscopic systems. Specifically, it is designed to model layered semiconductor heterostructure devices with arbitrary surface gate geometries. The electron density and electrostatic potential are calculated using self-consistent iterative procedure. The potential is calculated using finite-element analysis and the density is (typically) calculated using density functional theory. Simulations in 1, 2 and 3-D can be performed with varying degrees of approximation for the density. Dopents and surface charges are included. We intend the solutions of QuMESHS to be accurate representations of the actual experimental devices without the need to resort to using model potentials.

QuMESHS is written in C\#. The density calculations and the control code is performed in the main body of the QuMESHS code. For 1D calculations, QuMESHS can also calculate the electrostatic potential but for 2 and 3-D, we use external programs to calculate the electrostatic potential using the finite-element method. At the moment, this is done using either FlexPDE or the deal.II FEA library but QuMESHS can use any external program to generate these potentials.

QuMESHS was designed to model electrons in semiconductor heterostructures. The original focus was to study quasi-one-dimensional constrictions, with a specific focus on the low density limit, but the software has been designed to be applicable to more general semiconductor heterostructure systems.

QuMESHS can simulate 1, 2 and 3-D electron and hole systems in layered semiconductor heterostructures.  The software is still under development and it is recommended that if you want to use QuMESHS you get in contact with us either through www.qumeshs.org or by emailing eto24@cam.ac.uk.

QuMESHS was developed by Edmund T. Owen and Crispin H. W. Barnes with additional contributions by Thomas Gemunden. This work was developed under the EPSRC grant ``Nanoelectronic Based Quantum Physics - Technology and Applications'' under the direction of Mike Pepper and supported by the Engineering and Physical Sciences Research Council (EPSRC), UK, EP/K004077/1.


\section{Setting up QuMESHS}
\label{sec:Setup}

QuMESHS is a console application and must be run through the command line.  The executable takes two console input arguments.  The first argument is an unsigned integer which should, by default, be set to 0.  If you are using QuMESHS batch run system, then this is the batch run ordinal (see Sec.~\ref{subsubsec:BatchRuns} for details).  The second argument is the string containing the location of the input parameter file as described in Sec.~\ref{subsec:InputFile}.

\subsection{Windows}
\label{subsec:SetupWindows}

QuMESHS can either be run on Windows using the compiled binaries or by compiling the code yourself.  In order to compile QuMESHS, you must have a license for the NMath library provided by CenterSpace.  Please see the accompanying README for details.

With the binaries and libraries provided in the release contained within the same folder and with correctly formatted input files (see Sec.~\ref{sec:Inputs}), open a PowerShell in this folder and run the executable \texttt{Solver\_Master.exe}. 

\subsection{Linux}
\label{subsec:SetupLinux}

QuMESHS can be run in Linux but, as it is written using C\#, QuMESHS requires mono in order to provide the correct runtime framework.  We suggest bundling the libraries provided in the release section of the QuMESHS repository using the mono command mkbundle.  Exact details on how this can be done can be found on the mono website or, if you are having problems, contact eto24@cam.ac.uk.

\subsubsection{Using deal.II}
\label{subsubsec:UsingdealII}

In Linux, the recommended way of calculating the electrostatic potentials is using deal.II.  At the moment, we have not provided any information on how to do this.  Contact eto24@cam.ac.uk for details.

\section{Editing the inputs}
\label{sec:Inputs}

There are three input files: The config file, which contains the QuMESHS configuration;
the input file, which contains the simulation input parameters; and the band structure
file, which contains the layered device band structure and geometry.  The comment key
for all of these files is \texttt{\#}.  All lines beginning with \texttt{\#} will be
discarded.  The format for config and input files is that declarations begin with
\texttt{\%} and consist of a key, an equals sign and the value.  Spacing is not important.
The band structure file has a slightly more restrictive syntax which is described in
sec~\ref{subsec:BandStructureFile}.

{\color{red} NOTE:} \emph{Key names and string values in the input files are case
sensitive!  Boolean values are not case sensitive but must be fully spelt.}

\subsection{The Config file}
\label{subsec:ConfigFile}

The configuration file contains all the information which QuMESHS needs in order
to run.  Device parameters and simulation domain geometry should not be stored here
(although it can be as long as it isn't duplicated in the \texttt{Input\_Parameters.txt}
file).  This is mostly for defining where the software to generate the potentials is
located.  This file should be edited when installing QuMESHS on a computer and
(hopefully) never altered again.  {\color{red} NOTE:} \emph{Presently, the
configuration file must be called} \texttt{Solver\_Config.txt} \emph{and located in
the folder where QuMESHS runs!}

\paragraph{\texttt{tolerance} \emph{(compulsory)}}
Tolerance of the solution.  This is the maximum change in the chemical potential
across the entire solution from the Newton step $x$ which QuMESHS wanted to add
for which the solution is considered converged.  {\color{red} NOTE:} This is not
the chemical potential added at the final iteration step as it does not include
the mixing parameter $t$.  This the chosen definition for convergence as it is
related to the residual $g(\phi) = - \nabla \cdot ( \epsilon \nabla \phi ) -
\rho (\phi)$ through the Newton step $g' (\phi) x = - g (\phi)$ rather than the
optimised potential $t x$ which is what is actually added to the solution each
iteration step.  It is in units of meV.  

\paragraph{\texttt{tolerance\_1d} \emph{(compulsory)}}
Same as \texttt{tolerance} but for the 1D band structure calculation.  If
\texttt{dim = 1}, this is the tolerance used to calculate the initial dopent
density.

\paragraph{\texttt{use\_FlexPDE\_1d} \emph{(compulsory for 1D)}}
An override for the \texttt{Solver\_Config.txt} file in order to stop having
to change the \texttt{use\_FlexPDE} option when changing dimensions.  FlexPDE
is currently not available for 1D so whilst you probably need to use it for
2- and 3-D calculations, \texttt{use\_FlexPDE} should probably be set to
\texttt{true}.  However, this would prevent the 1D calculations from working
so setting with this override \texttt{Solver\_Config.txt} does not need to be
edited.  Default is \texttt{false} (and should always be).

\paragraph{\texttt{BandStructure\_File} \emph{(compulsory)}}
Location of the file containing the band structure information (see
sec.~\ref{subsec:BandStructureFile}).

\paragraph{\texttt{use\_FlexPDE} \emph{(compulsory)}}
Boolean to determine whether to use FlexPDE to calculate electrostatic potentials.
For 2- or 3-D, either this option or \texttt{use\_deal.II} must be true.  See
http://www.pdesolutions.com for details.

\paragraph{\texttt{FlexPDE\_file} \emph{(compulsory for FlexPDE)}}
Name of the FlexPDE script file (which should end in .pde although this might
not be necessary).  At the moment, QuMESHS will generate a custom FlexPDE
script and this is where this file will be written.  In future, we could try
and allow custom FlexPDE scripts which QuMESHS can edit.  The intermediate
output from a FlexPDE solution will be written to a file with the same name but
with the .pg6 extension.

\paragraph{\texttt{FlexPDE\_location} \emph{(compulsory for FlexPDE)}}
Location of the FlexPDE executable in the file system.  This should be an absolute
path.  QuMESHS has been extensively tested with FlexPDE6 and any backwards
compatibility with earlier versions of FlexPDE may be unstable.

\paragraph{\texttt{use\_deal.II} \emph{(compulsory)}}
Boolean to determine whether to use the \texttt{deal.II} library to calculate
electrostatic potentials.  For 2- or 3-D, either this option or
\texttt{use\_FlexPDE} must be true.  For a fuller explanation of how to
run QuMESHS with deal.II, see sec~\ref{subsubsec:UsingdealII}.  Also,
see https://dealii.org/ and associated sites for details.

\paragraph{\texttt{initcalc\_location} \emph{(compulsory with deal.II)}}
Location of the pre-compiled \texttt{deal.II} executable for calculating the initial
potential distribution $\phi$.  Whilst the dimensions of the device are flexible
(as given by \texttt{initcalc\_parameterfile}), the geometry for deal.II
must be defined and compiled separately.  See sec~\ref{subsubsec:UsingdealII}
for details.

\paragraph{\texttt{newton\_location} \emph{(compulsory with deal.II)}}
Location of the pre-compiled \texttt{deal.II} executable for calculating the Newton
step solution $x$.  Whilst the dimensions of the device are flexible
(as given by \texttt{initcalc\_parameterfile}), the geometry for deal.II
must be defined and compiled separately.  See sec~\ref{subsubsec:UsingdealII}
for details.

\paragraph{\texttt{initcalc\_parameterfile}}
Location of the parameter file with the inputs for the initial potential
calculation.  Note that, at the moment, this is defunct and QuMESHS will
overwrite this file with the parameters defined by ``\texttt{Input\_Parameters}''.

\paragraph{\texttt{newton\_parameterfile}}
Location of the parameter file with the inputs for the Newton step
calculation.  Note that, at the moment, this is defunct and QuMESHS will
overwrite this file with the parameters defined by ``\texttt{Input\_Parameters}''.

\paragraph{\texttt{with\_checkpointing}}
For some runs (especially batch runs on clusters), QuMESHS might be terminated
before a converged solution has been found.  This option allows for checkpoint
recovery.  All necessary data is always output (see sec~\ref{subsec:CheckpointOutput})
but for QuMESHS to restart from the last iteration step, this option must be
\texttt{true}.

\paragraph{\texttt{dft\_mixing\_parameter}}
QuMESHS uses function mixing for calculating a self-consistent DFT potential
in 2D and 3D (rather than the Bank-Rose/Newton-Raphson method), that is
$V_{n+1} = (1 - \alpha) V_{n} + \alpha V_{\mathrm{calc}}$.  This parameter is
the mixing factor $\alpha$.  Default is $\alpha = 0.3$.


\subsection{The Input File}
\label{subsec:InputFile}

The input file contains all of the simulation parameters QuMESHS needs.  These will
be different depending on the dimensionality of the system to be simulated but there
are some universal parameters which must always be specified for QuMESHS to run.

\paragraph{\texttt{dim} \emph{(compulsory but not yet implemented\ldots)}}
Dimensionality of the problem you want QuMESHS to solve.

\paragraph{\texttt{T} \emph{(compulsory)}}
The temperature in K for which the density will be calculated.  For the 1-D solutions,
this is the base temperature (see sec~\ref{sec:Overview} and sec~\ref{sec:DopentTheory}
for details).

\paragraph{\texttt{nz} \emph{(compulsory)}}
Number of lattice points for the growth direction.  Note that for 2- and 3-D solutions,
this is mostly used along with \texttt{dz} to define the domain size (see sec~\ref{?}).

\paragraph{\texttt{dz} \emph{(compulsory)}}
Lattice point spacing in nm for the growth direction.  Note that for 2- and 3-D
solutions, this is mostly used along with \texttt{nz} to define the domain size
(see sec~\ref{?}).

\paragraph{\texttt{nz\_dens}}
Number of lattice points for the growth direction on which to calculate the density.

\paragraph{\texttt{dz\_dens}}
Lattice point spacing in nm for the growth direction on which to calculate the density.

\paragraph{\texttt{zmin\_dens}}
Location in nm for the growth direction of the lattice on which to calculate the density.

\paragraph{\texttt{dft}}
Boolean to determine whether to use density functional theory or not.  The default is
\texttt{true}.

\paragraph{\texttt{no\_dft}}
Inverse of \texttt{dft}.

\paragraph{\texttt{TF\_only}}
Boolean to determine whether to only calculate a semi-classical, Thomas-Fermi density
distribution (see~\ref{sec:DensityTheory}).  The default is \texttt{false}.

\paragraph{\texttt{surface\_check}}
By default, QuMESHS will throw an exception if the top of the 1D band structure is not at
the surface.  If this parameter is set to \texttt{true}, then this check is ignored.
{\color{red} NOTE:} This means that typical 1D simulations never include the PMMA or air
above a sample.

\paragraph{\texttt{bottom\_V}}
Voltage at the base of the band structure in V.  By default, for 1D simulations, this is
the equilibrium chemical potential of the material at the base of the structure for the
given band gap, temperature and dopent densities given there.

\paragraph{\texttt{top\_V}}
Voltage at the top of the band structure in V.  By default, this is set to zero (see
sec~\ref{?} for explanation).

\paragraph{\texttt{illuminated}}
Whether the dopents are illuminated or not.  QuMESHS treats illumination by ionizing all
of the dopents.  When calculating the 1D band structure, the dopents ``freeze out'' at
a predefined temperature (70K in the case of AlGaAs).  This fixes the proportion of the
dopents ionized at lower temperatures as not all of the dopents will be ionized at
freeze-out.  Illumination ionizes all of the dopents and changes the dopent distribution
to that defined in the band structure input file.  This typically increases the carrier
density but is generally an underestimate as QuMESHS does not include ``deep'' donors.

\paragraph{\texttt{output\_suffix}}
Suffix for output data files.  In sec~\ref{subsec:DataOutput}, this is the string which
replaces \texttt{\_<suffix>}. {\color{red} NOTE:} This will be overridden if running
a batch job (see sec~\ref{subsubsec:BatchRuns}).

\paragraph{\texttt{max\_iterations}}
Number of iterations to try before giving up and outputting the data how it is.  The
default value is 1000.  If the solution has not converged, a file beginning with
\texttt{not\_converged\_<suffix>} will be included in the output (see sec~\ref{sec:Output}).

\paragraph{\texttt{initial\_run}}
This is a boolean which sets whether QuMESHS will do an initial run in order to get a
better initial density distribution.  Typically, the density for the first potential
calculation is not a very good approximation to the actual density so there are lots of
instabilities in the initial few steps.  It can take a long time for QuMESHS to get rid
of these so one work-around is to let QuMESHS make a few steps and then use this distribution
as an initial input.  Hopefully, this density is better and so the solution will converge quicker.
Default is set to \texttt{true}.

\paragraph{\texttt{initial\_run\_steps}}
This is the number of steps which QuMESHS will iterate before starting again based on
the protocol explained above.  Only used if \texttt{initial\_run = true}.  Default is
5.

\paragraph{\texttt{carrier\_type}}
The type of carriers which the solution will calculate for.  In 2D and 3D, this is either
``electron'' or ``hole''.  Combining them has now been implemented.  In 1D, it is also possible
to specify that you want to allow ``both'' although it is recommended that you use either
``electron'' or ``hole'' if you know what you expect.

\commentout{
\subsubsection{1D}

\red{I cannot think of any 1D specific input parameters\ldots}
}

\subsubsection{2D}

In order to calculate the dopent density distribution for the 2D calculations, a 1D
solution in the growth direction must be calculated first.  The parameters are still
defined in the ``\texttt{Input\_Parameters.txt}'' file but with the suffix ``\texttt{\_1d}''.
ie. \texttt{nz\_1d} and \texttt{dz\_1d} whilst the optional inputs are \texttt{nz\_dens\_1d},
\texttt{dft\_1d}, etc\ldots  The only extra parameter for the 1D calculation is
\texttt{ny\_1d} which gives the number of lattice sites to replicate the dopent distribution
for FlexPDE calculations.

\paragraph{\texttt{ny} \emph{(compulsory)}}
Number of lattice points for the direction transverse to the transport direction.  Note
that this is mostly used along with \texttt{dz} to define the domain size).

\paragraph{\texttt{dy} \emph{(compulsory)}}
Lattice point spacing in nm for the direction transverse to the transport direction.
Note that this is mostly used along with \texttt{nz} to define the domain size).

\paragraph{\texttt{ny\_dens}}
Number of lattice points for the direction transverse to the transport direction on which
to calculate the density.

\paragraph{\texttt{dy\_dens}}
Lattice point spacing in nm for the direction transverse to the transport direction on
which to calculate the density.

\paragraph{\texttt{ymin\_dens}}
Location in nm for the direction transverse to the transport direction of the lattice on
which to calculate the density.

\paragraph{\texttt{voltages} \emph{(compulsory)}}
An array of the voltages applied to all of the gates defined in the band structure file.
the array is enclosed by braces and comma delimited eg. \texttt{\{ -1.0, -1.5 \}}.  The
ordering of the gates is defined by the band structure file and is bottom up in terms of
layers and then first to last for the composite layers which contain the gates.

\paragraph{\texttt{surface\_charge}}
By default, the charge on the surface of the device is calculated from the 1D dopent
calculation in order to pin the surface mid-gap for cool-down when all gates are grounded
(see sec~\ref{subsec:PotentialSolvers} for explanation).  However, it is possible to
override this value here.

\paragraph{\texttt{split\_width} \emph{(compulsory)}}
QuMESHS was originally written to study split gates in 2D.  This parameter defines the
width of the split.  \red{Whilst the split width should be defined entirely by the band
structure file, QuMESHS is not currently equipped to get at this information\ldots}

\paragraph{\texttt{pot\_tol}}
Tolerance for FlexPDE for the calculation of the initial potential distribution.
This is the ``\texttt{ERRLIM}'' setting value.  Default is 1E-06.

\paragraph{\texttt{newton\_tol}}
Tolerance for FlexPDE for the calculation of the Newton step potential.
This is the ``\texttt{ERRLIM}'' setting value.  Default is 1E-05.

\paragraph{\texttt{density\_solver\_2d}}
Gives the level of approximation for calculating the 2D density.  The options are:
\begin{enumerate}
    \item effectiveband
    \item thomasfermi
    \item dft
    \item sodft (untested)
\end{enumerate}
Further details of the theory of these approximations is given in Sec.~\ref{sec:Theory}.


\subsubsection{3D}

In order to calculate the dopent density distribution for the 3D calculations, a 1D
solution in the growth direction must be calculated first.  The parameters are still
defined in the ``\texttt{Input\_Parameters.txt}'' file but with the suffix ``\texttt{\_1d}''.
ie. \texttt{nz\_1d} and \texttt{dz\_1d} whilst the optional inputs are \texttt{nz\_dens\_1d},
\texttt{dft\_1d}, etc\ldots  The only extra parameters for the 1D calculation are
\texttt{nx\_1d} and \texttt{ny\_1d} which gives the number of lattice sites to replicate
the dopent distribution for FlexPDE calculations.

\paragraph{\texttt{nx} \emph{(compulsory)}}
Number of lattice points for the direction along the transport direction.  Note
that this is mostly used along with \texttt{dz} to define the domain size.

\paragraph{\texttt{dx} \emph{(compulsory)}}
Lattice point spacing in nm for the direction along the transport direction.
Note that this is mostly used along with \texttt{nz} to define the domain size.

\paragraph{\texttt{nx\_dens}}
Number of lattice points for the direction along the transport direction on which
to calculate the density.

\paragraph{\texttt{dx\_dens}}
Lattice point spacing in nm for the direction along the transport direction on
which to calculate the density.

\paragraph{\texttt{xmin\_dens}}
Location in nm for the direction along the transport direction of the lattice on
which to calculate the density.

\paragraph{\texttt{ny} \emph{(compulsory)}}
Number of lattice points for the direction transverse to the transport direction.  Note
that this is mostly used along with \texttt{dz} to define the domain size.

\paragraph{\texttt{dy} \emph{(compulsory)}}
Lattice point spacing in nm for the direction transverse to the transport direction.
Note that this is mostly used along with \texttt{nz} to define the domain size.

\paragraph{\texttt{ny\_dens}}
Number of lattice points for the direction transverse to the transport direction on which
to calculate the density.

\paragraph{\texttt{dy\_dens}}
Lattice point spacing in nm for the direction transverse to the transport direction on
which to calculate the density.

\paragraph{\texttt{ymin\_dens}}
Location in nm for the direction transverse to the transport direction of the lattice on
which to calculate the density.

\paragraph{\texttt{split\_length} \emph{(compulsory?)}}
QuMESHS was originally written to study split gates with a bar gate on top in 3D.  This
parameter defines the length of the split gates (ie. in the transport direction).
\red{Whilst the split gate length should be defined entirely by the band structure file,
QuMESHS is not currently equipped to get at this information\ldots}

\paragraph{\texttt{top\_length} \emph{(compulsory?)}}
QuMESHS was originally written to study split gates with a bar gate on top in 3D.  This
parameter defines the length of the bar gate on top (ie. in the transport direction).
\red{Whilst the top gate length should be defined entirely by the band structure file,
QuMESHS is not currently equipped to get at this information\ldots}

\paragraph{\texttt{initialise\_with\_1d\_data}}
For most of the 3D density calculations, the charge distribution in the growth direction
is factored out (see sec~\ref{subsec:DensityTheory} for details).  Therefore, a good
initial guess at the charge distribution is that it is the original 2DEG solution (ie.
with the surface voltage set to zero).  This should accelerate convergence to the solution.
This parameter determines whether this should be done.  By default, it is set to
\texttt{false} and the initial solution is calculated with no charge in the well.


\subsubsection{Batch Runs}
\label{subsubsec:BatchRuns}

QuMESHS allows the user to run a batch of simulations with different boundary
conditions which is useful for parameter sweeps.  When running a batch job, each
simulation will be assigned a simulation number.  This is counted from zero and can be
offset by an integer value with the first console argument of QuMESHS.  Therefore,
if the batch run is managed externally (using Condor, say), then the number of simulations
should be set to one and the first console argument is provided by the batch run manager
(see Sec.~\ref{sec:Setup}).
\texttt{batch\_run} must still be set to true, otherwise the boundary conditions will
not be incremented according to your batch run definitions.

Batch run definitions are given below.  In essence, the parameters which you want
to increment must be described with the \texttt{batch\_params} argument.  The given
parameters will then be incremented in order on a hypercuboid lattice of size
$\prod_i$ \texttt{no\_<param>}$_i$.  For the concrete example given for
\texttt{batch\_params}, the \texttt{top\_V} parameter will be incremented from
zero to \texttt{no\_top\_V - 1} and then \texttt{split\_V} will be incremented by
one and \texttt{top\_V} will again be incremented from zero to \texttt{no\_top\_V
- 1}, etc., etc..  {\color{red} NOTE:} It is possible to increment the \texttt{voltages}
option in the input file but the \texttt{batch\_params} argument must be the
specific voltage values (eg. \texttt{V0}, \texttt{V1}, etc.)

\paragraph{\texttt{batch\_run}}
Boolean for checking whether or not to run a batch of runs.  \texttt{false} by default.

\paragraph{\texttt{no\_sims}}
Number of simulations to put in the batch.  Default is one.

\paragraph{\texttt{batch\_params}}
This is a comma-delimited array of the parameters which will be incremented in the
order in which they will be incremented (see beginning of section).  QuMESHS will
check to make sure that the relevant initial values, the increment value and the
number of times to increment that value are specified in the input file.  An example
format is \texttt{\{top\_V, split\_V\}}, taken from the example in
sec~\ref{subsec:TopAndSplitGate}, where the braces (which aren't strictly necessary)
will be trimmed.  For the following input option definitions, \texttt{<param>} is the
string defined in \texttt{batch\_params}.  There is also the option of setting the
\texttt{voltages} labels to be equal.  For example, to set \texttt{V1=V0} whilst
incrementing both, the appropriate format is \texttt{\{V0, V1=V0\}}.  {\color{red} NOTE:}
Which voltages are not determined by the batch number must be the final entries.

\paragraph{\texttt{delta\_<param>}}
Step size for the increments of \texttt{<param>} for batch runs.

\paragraph{\texttt{no\_<param>}}
Number of times to increment \texttt{<param>} by \texttt{delta\_<param>}.

\paragraph{\texttt{init\_<param>}}
Initial value of \texttt{<param>} to step from.


\subsection{The Band Structure File}
\label{subsec:BandStructureFile}

The band structure is defined in a file as a set of layers (in the growth direction), each
defined on a single line.  The default geometry of a layer is a slab of a given thickness.
However, composite layers can be constructed with different materials in more complex
geometries (as described below).  Each layer has a set of options -- some compulsory --
and there are also special layer commands to define the surface and substrate.  The
ordering of the layer options is unimportant.

\subsubsection{Layer Options}
\label{subsubsec:LayerOptions}

{\color{red}IMPORTANT:} \emph{Each layer option consists of the option, an equals sign and the value.
There MUST be no spaces between these, otherwise the options will not be input correctly}.
ie. \texttt{mat=GaAs} will work but \texttt{mat = GaAs} or \texttt{mat= GaAs} will not.

\paragraph{\texttt{mat} \emph{(compulsory)}}
Material from which the layer is made.  The band gap and permittivities for each of the
materials are given in Appendix~\ref{app:MaterialProperties}.  Layer-specific spin-orbit
parameters are under development.   Options are:

\begin{itemize}
    \item{GaAs}
    \item{Al\textsubscript{x}Ga\textsubscript{1-x}As}
    \item{PMMA}
    \item{Air}
    \item{Metal}
    \item{In\textsubscript{x}Ga\textsubscript{1-x}As}
    \item{In\textsubscript{x}Al\textsubscript{1-x}As}
\end{itemize}

\paragraph{\texttt{t} \emph{(compulsory)}}
Thickness of the layer in nm.  Note that the absolute position of the layers is calculated
by QuMESHS based on the \texttt{surface=true} definition described
in~\ref{subsubsec:SpecialLayers}.

\paragraph{\texttt{x} \emph{(compulsory for alloys)}}
Alloy composition variable as a decimal fraction (eg. 33\% AlGaAs is defined by \texttt{x=.33}).
This is used to calculate the band gap and permittivity of the alloy.  See
Appendix~\ref{app:MaterialProperties} for details.

\paragraph{\texttt{Na}}
Number density of acceptors in cm\textsuperscript{-3}.  Default is zero.

\paragraph{\texttt{Nd}}
Number density of donors in cm\textsuperscript{-3}.  Default is zero.

\subsubsection{Composite Layers}

When describing geometries more complex than layered slabs, the ``composite layer'' format
must be used.  At the moment, most implementations are used to define surface gates.  An
example layer is given by
\begin{center}
\texttt{mat=PMMA t=10 composite=true no\_components=3 \{mat=metal,geom=half\_slab,dx=0.0,dy=-400.0,theta=90.0\} \{mat=metal,geom=half\_slab,dx=0.0,dy=400.0,theta=270.0\}}
\end{center}
{\color{red} NOTE:} In the band structure file, all of these options \emph{must} be on the
same line.  The first two entries describe the parameters of the default layer.  Any part
of the layer which is not specifically defined will have the property of this material
(PMMA in this example).  The thickness of the layer is also described here and applies to
all components of the layer.

\paragraph{\texttt{composite} \emph{(compulsory for composite layers)}}
Boolean flag to indicate that this line defines a composite layer.  If this is set to
\texttt{false} then the band structure will generate the default layer.

\paragraph{\texttt{no\_components} \emph{(compulsory for composite layers)}}
Indicates how many components make up this layer.  Note that this includes the default
layer.  If this is less than the defined number of components, then the behaviour is
undefined but QuMESHS will probably just discard the trailing component definitions.

\vspace{0.5cm}

The layer descriptions in the braces define the components of the layer.  {\color{red} NOTE:}
There are no spaces and the options are comma delimited. \emph{This is compulsory formatting
and incorrect band structures will be generated if there are spaces within the braces!}  The
end brace \} is not strictly necessary.  Each component must have a \texttt{mat} option.  The
rest of the options describe the geometry of the component layer.  How these parameters
relate to the components is described in Appendix~\ref{app:GeometryDefinitions}.  The
application of arbitrary layers is very incomplete and mostly only functional for strips
or half-slaps parallel to either the $x$ or $y$ axes.

\subsubsection{Special Layers}
\label{subsubsec:SpecialLayers}

\paragraph{\texttt{surface=true} \emph{(compulsory)}}
This indicates the location of the surface.  The surface is always positioned at $z = 0$ and
all interface positions are calculated with respect to this.  All layers defined before the
\texttt{surface=true} have $z > 0$ and all layers defined after \texttt{surface=true} have
$z < 0$.

\paragraph{\texttt{mat=substrate} \emph{(compulsory)}}
This is a special material which must be defined at the end of the band structure file.  Any
layers below this line will be discarded.  This layer extends from whereever the bottom layer
ends to $-\infty$.  It is possible to set the boundary conditions on this interface manually
so if there is no substrate (for example, with a thinned, back-gated device) this layer is
still necessary but has no effect.


\section{Outputs}
\label{sec:Output}

\subsection{Console output}
\label{subsec:ConsoleOutput}

The setup output from the console should be relatively self-explanatory.  During the simulation,
the console will output data for each iteration:  The iteration number, labelled \texttt{Iter},
a measure of how much the density has changed from the previous step, labelled \texttt{Dens conv},
the calculated optimal value for the Newton mixing, labelled \texttt{t} and the time taken to
perform the current iteration in minutes, labelled \texttt{time}.  For the 1D simulations, the
time is not output.

For DFT calculations, the exchange-correlation potential is mixed in when the electrostatic
potential has converged to a ``good enough'' self-consistent solution (see
sec~\ref{sec:Theory}).  When this occurs, the console will say what the maximum
and minimum density difference between the current and calculated exchange-correlation potential
is.

At the end of the iteration, the console will say what the maximum change in the potential for
the final iteration was -- just to be sure or to check how ``not converged'' the solution is
if the maximum number of iterations has been reached.


\subsection{Converged data output}
\label{subsec:DataOutput}

QuMESHS outputs a variety of data from the simulation and the way it must be processed can
depend on the dimensionality of the simulation.  Unless otherwise specified, the data
files contain various functions defined on the density lattice given by the input file.
The format of the output suffix \texttt{<suffix>} is given in sec~\ref{subsec:InputFile}.

\paragraph{\texttt{bare\_pot\_<suffix>}}
This is the bare potential in meV for the system assuming that there is no carriers in
the density simulation domain.  Note that this is the quantum potential of the conduction
band and, therefore, includes the relevant band gaps.  This is the result of the initial
potential calculation with the input dopent density and gate voltages.

\paragraph{\texttt{dens\_<suffix>}}
This is the spin-summed carrier density in nm\textsuperscript{-3}.

\paragraph{\texttt{dens\_up\_<suffix>}}
This is the spin-up carrier density in nm\textsuperscript{-3}.

\paragraph{\texttt{dens\_down\_<suffix>}}
This is the spin-down carrier density in nm\textsuperscript{-3}.

\paragraph{\texttt{density\_error\_<suffix>}}
This is the difference between the final spin-summed density and the spin-summed density
if the simulation had continued for another iteration step.  If the simulation has
converged, then the values in this file should all be very small.

\paragraph{\texttt{energies\_<suffix>}}
These are the energies in meV of the eigenstates calculated from the final iteration step.
This may be one of the most important outputs of QuMESHS.  The eigenstates for solutions
in 1- and 2-D are the zero-momentum states (ie. the bottom of the bands).  The energy is
given with respect to the chemical potential $\mu$.

\paragraph{\texttt{pot\_KS\_<suffix>}}
This is the converged, self-consistent Kohn-Sham potential in meV felt by the electrons in
the system.  Note that this is the quantum potential of the conduction band and, therefore,
includes the relevant band gaps.  This is the Kohn-Sham potential and, therefore, includes
the exchange-correlation potential if the calculation includes DFT.

\paragraph{\texttt{potential\_<suffix>}}
This is the converged, self-consistent electrostatic potential in meV
for the system.  Note that this is the quantum potential of the conduction band and,
therefore, includes the relevant band gaps.  Also, this is the electrostatic potential
and, therefore, does not include the exchange-correlation potential if the calculation
includes DFT.

\paragraph{\texttt{<FlexPDE\_File>\_<suffix>.pg6} \emph{(if using FlexPDE)}}
FlexPDE outputs graphs when a potential or Newton step calculation is complete.  This
file will contain the final FlexPDE solution data (whatever that may be\ldots).

\paragraph{\texttt{surface\_charge\_<suffix>}}
This is the surface charge \commentout{\red{what units?}} calculated from the 1D dopent distribution
calculation at the beginning of the simulation.

\paragraph{\texttt{xc\_pot\_<suffix>}}
This is the exchange-correlation potential in meV for the converged density distribution.

\paragraph{\texttt{dens\_<dim>D\_dopents.dat}}
This is the dopent distribution used by FlexPDE to calculate the initial potential.  It
might be useful, but in general, this data is better acquired elsewhere.

\paragraph{\texttt{dopent\_1D.dat}}
The 1D dopent density distribution for deal.II.  It might be useful, but in general, this
data is better acquired elsewhere.

\paragraph{\texttt{not\_converged\_<suffix>}}
Hopefully not present, but if QuMESHS could not find a converged solution then this
file will be output with the maximum number of iterations that QuMESHS ran for.
{\color{red} NOTE:} All other output data files will still be saved even though the
solution did not achieve convergence.


\subsection{Observing output during calculations}
\label{subsec:IntermediateOutput}

Sometimes it is useful to see what QuMESHS is doing without interrupting a calculation.
Unfortunately, this ability is very poorly supported.  The best to offer is the intermediate
solutions of FlexPDE.  These are provided in .pg6 files with the prefix given by the
\texttt{FlexPDE\_file} parameter in the config file.  These are updated regularly (ie.
at every iteration step) but if the size of the file is 0KB then the simulation is in
progress and you need to wait until it has finished.

\texttt{deal.II} simulations are for batch runs and there are currently no convenient ways
of getting at the data.  However, \texttt{deal.II} can output files to \texttt{vtk} format
easily so it is possible to output data by editing the source code and recompiling the
\texttt{deal.II} software.


\subsection{Outputs for Checkpointing}
\label{subsec:CheckpointOutput}

QuMESHS has the capability of restarting a simulation from a given iteration step if asked
to do so.  This enables a run to continue even if the simulation ended before the solution
converged.  This is useful for running on clusters where QuMESHS might only have a limited
time to execute without losing all its progress.  In order to enable this, the configuration
file must include the parameter \texttt{with\_checkpointing = true}.

The data required to restart a simulation is output every time step (just in case).  The
relevant files are \texttt{carrier\_density.tmp}, \texttt{chem\_pot.tmp}, \texttt{dopent\_density.tmp},
\texttt{t\_val.tmp} and \texttt{restart.flag}.  If the simulation converges or the maximum
number of iterations is reached, these checkpointing files are automatically deleted.  Also,
if \texttt{restart.flag} is deleted, then checkpointing will not restart.  This allows the
user to prevent QuMESHS from restarting from a checkpoint without changing the config file.


\section{Theory}

QuMESHS solves the self-consistent Poisson-Schr\"{o}inger equation in layered semiconductor
heterostructures using a variety of approximate methods to calculate the electron and
hole densities.

\subsection{Poisson's Equation}

The first of the two coupled equations which we want to solve is the Poisson equation
%
\begin{equation}
    \label{eq:Poisson}
    g \left[ \phi(r), \rho(r) \right] \equiv - \nabla \cdot \left(\epsilon_0 \epsilon(r) \nabla \phi(r)\right) - \rho(r) = 0
\end{equation}
%
where $\epsilon_0$ is the permittivity of free space and $\epsilon(r)$ is the relative
dielectric constant, which is different for the various materials in the heterostructure.
$\phi(r)$ is the electrostatic potential and $\rho(r)$ is the electric charge.   The
function $g$ is positive-definite.

QuMESHS solves Eq.~\ref{eq:Poisson} using the finite-differences method for 1D and
the finite-element method for 2D and 3D.  The finite-element method is better explained
elsewhere in the respective documentation for FlexPDE and \dealii but we will explain
the finite-differences method here.  In order to solve Eq.~\ref{eq:Poisson}, the
Laplacian $\nabla \cdot \left(\epsilon_0 \epsilon(r) \nabla \phi(r)\right)$ needs to
be discretised onto a lattice.  QuMESHS only implements finite
differences on a regular grid in 1D.  However, here we show how to discretise the
Laplacian on an irregular grid.

In 1D, we want to discretise
%
\begin{eqnarray}
    \frac{\partial}{\partial z} \left( \epsilon_0 \epsilon(z) \frac{\partial \phi(z)}{\partial z} \right)
    & = & \frac{\epsilon_0}{(\Delta_- + \Delta_+) / 2} \left( \epsilon(z + \Delta_+ / 2) \left. \frac{\partial \phi(z)}{\partial z} \right|_{z + \Delta_+ / 2}
    - \epsilon(z - \Delta_- / 2) \left. \frac{\partial \phi(z)}{\partial z} \right|_{z - \Delta_- / 2} \right) \\
    & \approx & \frac{\epsilon_0}{(\Delta_- + \Delta_+) / 2} \left(\epsilon(z + \Delta_+ / 2) \frac{\phi(z + \Delta_+) - \phi(z)}{\Delta_+}
    - \epsilon(z - \Delta_- / 2) \frac{\phi(z) - \phi(z - \Delta_-)}{\Delta_-} \right) \\
    & = & \frac{\epsilon_0}{\Delta^2} \left(\epsilon(z + \Delta_+ / 2) \left(\phi(z + \Delta_+) - \phi(z)\right) -  \epsilon(z - \Delta_- / 2) \left(\phi(z) - \phi(z - \Delta_-)\right) \right)
\end{eqnarray}
%
where the final equality is for $\Delta_- = \Delta_+$, ie. a regular grid (as used in
QuMESHS). \commentout{ This discretisation can be put in matrix form:
%
\red{some matrix}
%
\red{some details on how to do boundary conditions}}
This allows us to reduce Eq.~\ref{eq:Poisson} to a matrix equation where
%
\begin{equation}
    A \phi = - \rho
\end{equation}
%
which is solved for a known $\rho$ by $LU$ factorisation.


\subsection{Density}

Calculating the density for the many-body problem is hard~\cite{:-P}.  The equation
to solve is the $N$-body Schr\"{o}dinger equation:
%
\begin{equation}
    \left( \sum_i \left[ \frac{-\hbar^2}{2 m_*} \nabla^2_i + V_{\mathrm{ext}} (r)\right] + \frac{e^2}{8 \pi \epsilon_0 \epsilon (r)} \sum_{i \neq j} \frac{1}{|r_i - r_j|} \right) \Psi = E \Psi
\end{equation}
%
where $\Psi = \Psi (r_1, \ldots, r_N)$ is appropriately anti-symmetrised.  $m_*$
is the effective mass of the electrons and $V_{\mathrm{ext}}$ is any externally
generated electrostatic potential (ie. from metallic gates).  The density is then
defined as
%
\begin{equation}
    \rho (r) = -e N \int \dd^3 r_2 \ldots \int \dd^3 r_N \Psi^* (r, r_2, \ldots, r_N) \Psi(r, r_2, \ldots, r_N)
\end{equation}
%
In reality, we will never be able to solve this set of equations so we must find
approximations for the charge density $\rho (r)$.

\commentout{
\subsubsection{Thomas-Fermi semi-classical approximation}

The first approximation is the simplest.  We assume that the electrons have a density
at $r$ which is equivalent to what the bulk density would be for the chemical potential
at this point $\mu (r)$.
}

\commentout{
\section{Code Layout}

QuMESHS uses the NMath library for a significant amount of its data manipulation and
eigenvector decomposition routines.  In order to recompile the code, the user will need
a copy of the NMath license (see http://www.centerspace.net/).


\subsection{Processing Inputs and Solver Setup}

QuMESHS has been adapted for numerous batch and configuration set-ups in its history and,
at the moment, the input routines can look a bit muddled in the code.  Essentially, anything
in the individual project \texttt{Program.cs} files should be checked if edits are being made
as none of this is very modular at the moment.  The general structure involves getting the
input parameters and band structure from the relevant text files and calculating the dopent
density distribution by running a 1D simulation.  Note that for 2- and 3-D calculations using
FlexPDE, the scaling of the dopent density is performed here (see sec~\ref{subsubsec:UsingFlexPDE}).

Once one of the \texttt{Experiment} class is created and a suitable input dictionary has been
generated, the behaviour of QuMESHS is more predictable.  Each \texttt{Experiment} class
runs a \texttt{Initialise\_Experiment} routine which uses \texttt{Experiment\_Base} to initiate
the dimension-ambiguous parameters and then sets the remaining parameters.  Also, note that
the potential calculator is setup here.

There are a number of checks which \texttt{Experiment} can make at this point:
%
\begin{itemize}
    \item Check that you haven't asked for both FlexPDE and \texttt{deal.II} to be used
    \item Check that there is a band structure input file.
    \item Check that, for the window where the density is to be calculated, there is a lattice
    point on every boundary.  This is important as the Laplacian of the potential will not be
    calculated accurately if this is not true.
\end{itemize}

By this point, the \texttt{Experiment} has been set up and the \texttt{Run} method generates
a bare potential (if we have not started from a checkpoint).  One of the density solvers is
initiated here: \red{At the moment, changing the density solvers involves changing the source
code and recompiling\ldots this will be changed in future updates of QuMESHS}.  If the
initial carrier density is not zero, the potential is recalculated here to get the initial
potential.

An initial DFT mixing parameter is set (see sec~\ref{subsec:IterationRoutine}) and the DFT
potential for the initial charge distribution (which is normally zero) is calculated.  If DFT
is included, experience shows that it is better to include some DFT potential at the start so
the charge distribution is recalculated and the DFT potential for this charge is the initial
DFT potential $V_{\mathrm{xc}}$.


\subsection{Running the Iteration Routine}
\label{subsec:IterationRoutine}

The iteration routine can be broken down into seven steps

\begin{enumerate}
    \item Save previous density $\rho_{n-1}$.
    \item Calculate the new density distribution $\rho_n (\phi_n)$.
    \item Calculate $\rho'_n (\phi_n)$ for the Jacobian $g'(\phi_n)$.
    \item Calculate the Newton step function $x_n$ using the \texttt{IPois\_Solv} to solve
    $g'(\phi_n) x_n = - g(\phi_n)$.
    \item Calculate the optimal proportion $t_n$ of $x_n$ to add using the Bank-Rose technique.
    \item Check whether convergence has been achieved by evaluating $\delta\rho = \rho_n - \rho_{n-1}$.
    \item Update the potential $\phi_{n+1} = \phi_n + t_n x_n$.
\end{enumerate}

Calculating the density, its derivative and the Newton step function will be left to the following
sections.  Here, we will focus on the implementation of the Bank-Rose technique and also how the
convergence of the density is determined.

The Bank-Rose technique is a way of calculating the optimal proportion of $x_n$ -- the solution
to the Newton step equation~\ref{eq:NewtonStep} -- by finding the minimum of the function
%
\begin{equation}
    V'(t) = g^T (\phi + t x) \cdot x
\end{equation}
%
that is, you need to minimize the inner product between what the Newton step wants to add and the
residual of the non-linear equation.  This makes sense as the technique not only tries to reduce
$g(\phi + tx)$ but it also tries to do so in the regions where $x$ is largest (ie. where the
solution is going to be changed the most).  It is guaranteed that $V'(t)$ is a monotonic function
of $t$ with a single zero for $t > 0$~\cite{Bank:????}.

$V'(t)$ is minimised using a binary chop search.  The search starts from the value calculated for
the previous time step $t_0$.  We start by looking ``downhill'' by evaluating $V'(t_0)$ and
$V'(\alpha t_0)$ where $\alpha$ is some ``chopping'' parameter.  If the absolute value of $V'(t)$
is decreasing in this direction, then we keep chopping $t$ so that $V'(t)$ keeps getting smaller.
When $V'(t)$ goes through zero, we return the interpolated value of $t$ between $t_n$ and
$\alpha t_n$.  If the absolute value of $V'(t)$ was increasing after our
initial ``chop'' of $t_0$, we start increasing $t$ by the chopping parameter, ie. evaluate at
$t_0 / \alpha$.  At the moment, $\alpha = 0.8$ is hardcoded into the code in the
\texttt{Experiment\_Base} class\ldots \red{It is perfectly possible that this is a very poor
algorithm for finding this zero.}

An important comment here is that for most of the density calculations performed by QuMESHS,
the Jacobian for the density $\partial \rho_x (\phi) / \partial \phi_{x'}$ is approximated
locally for $x = x'$ only.  This means that the Newton step $x$ is only approximate and it is
possible that the function $V'(t)$ is not monotonic.  For this reason, the search is bounded
above by 1.0 and below by $t_{\mathrm{min}}$ which is currently hard-coded as $t_{\mathrm{min}}
= 10^{-3}$ in the \texttt{Experiment} classes\ldots

\red{should include an explanation of how convergence is assessed and how the exchange-correlation
potential is mixed in.}


\subsection{Potential Solvers}
\label{subsec:PotentialSolvers}

At the moment, QuMESHS supports three methods for calculating the initial potential and each
Newton iteration step.
%
\begin{itemize}
    \item Finite differences on a regularly spaced grid (1-D only)
    \item Finite element solution using FlexPDE (2- and 3-D)
    \item Finite element solution using \texttt{deal.II} (2- and 3-D)
\end{itemize}
%
The device geometries QuMESHS is designed to solve can be large with significant regions empty
of charge (and therefore described simply by the Laplace equation) with concentrated areas of
large charge densities.  This means that using finite differences on a regular grid is essentially
impossible for dimensions greater than 1D.  One of QuMESHS's advantages over other software
\red{(potentially NextNANO and SCRAPS)} is that it supports finite element libraries to solve
the relevant partial differential equations on irregular, adaptive meshes.

\red{It is perfectly possible to use finite element solutions for the 1D case.  This might even
speed up the convergence for calculating the dopent density distributions.  However, these are
probably fast enough so this has not been implemented.}

Calculating the initial potential and the Newton iteration step uses very similar geometries
and equations so these two steps are incorporated into \texttt{IPois\_Solv} classes.  The
boundary conditions for the initial potential calculations are exactly what you would expect
for solving Poisson's equation.  However, for the Newton iteration step, all boundary
conditions are set to zero.  This is because the initial potential $\phi_0$ already satisfies
all these boundary conditions so for $\phi_n = \phi_0 + \sum_n t_n x_n$ to satisfy these
boundary conditions too, the boundary conditions of $x_n$ must all be zero.  This includes
the surface and dopent charges as these are assumed to be frozen out so, by the linearity
of Poisson's equation, their potentials are also included in the initial potential $\phi_0$.

The boundary conditions for the potential solvers can be a tricky thing.  Commonly, when growing
semiconductor heterostructures, the material is lightly doped with acceptors.  Therefore, the
chemical potential at the base of the device might not be mid-gap.  By default, the boundary
condition at the base of the device will be set to the equilibrium chemical potential for the
doping at this point. This can be overriden by explicitly setting the \texttt{bottom\_V} option.

For the 1D calculations, it is assumed that the surface is pinned mid-gap.  This is an
experimental result.  Therefore, the boundary condition at the top of the domain (assuming the
\texttt{surface\_check} option has not been set to \texttt{false}) will be set to zero.  Once
the boundary value at the surface is set, this determines the surface charge.  Assuming that
the gates are grounded when the surface charge is frozen out and seeing as the device is
electrically neutral, the surface charge must be what is needed to bend the conduction band
so that it is flat outside the device.  The charge needed for this is calculated in the
\texttt{Get\_Surface\_Charge} method in \texttt{OneD\_PoissonSolver.cs}.  This surface
charge is then saved to the input dictionary which is used for 2- and 3-D calculations
of the initial potential.  {\color{red} NOTE:} At the moment, there is a depletion length
of the surface charge from the gates of 20nm~\cite{Laux:19??} for the 2D calculations in
both FlexPDE and \texttt{deal.II}.

The details of the finite differences solution \red{should have been} were addressed in
sec~\ref{subsec:PotentialTheory}.


\subsubsection{Using FlexPDE}
\label{subsubsec:UsingFlexPDE}

FlexPDE is a proprietary program for solving partial differential equations using
an adaptive, finite element method.  The software can be purchased on its website,
www.pdesolutions.com, and must be available for QuMESHS to calculate the potential
and Newton steps using FlexPDE.  This makes this method impractical for large scale
batch simulations (eg. parameter sweeps) but FlexPDE is useful for initial prototyping
due to its simple scripting language and easy-to-use interface.

The device description is written in a script file with a syntax described on the
website.  \red{At present, the only device geometry available in QuMESHS is for
a split gate on an arbitrary layered semiconductor heterostructure.  Future versions
of QuMESHS should allow the input of prototyped scripts where the relevant values
are input by QuMESHS.}

The input files for FlexPDE are input using the TABLE format.  {\color{red} NOTE:}
FlexPDE assumes that the values of the inputs at the edges of the TABLE extend
to the edge of the domain.  Therefore, it is essential that the edges are set to zero.
The dopent distribution is also input in the TABLE format so the initial 1D
calculation must be expanded into 2D.  This is done using the \texttt{Expand\_BandStructure}
method of the \texttt{Input\_Band\_Structure} class.  The dopent distribution is assumed
to be translationally invariant in the directions parallel to the surface.  The number
of sites in the $x$ and $y$ directions are defined in \texttt{Input\_Parameters.txt} as
the options \texttt{nx\_1D} and \texttt{ny\_1D}.


\subsubsection{Using deal.II}
\label{subsubsec:UsingdealII}

\texttt{deal.II} is a powerful open-source library designed for solving partial differential
equations using an adaptive, finite element method.  Full documentation can be found at
https://dealii.org/ and its associated website.  \emph{I cannot emphasise {\bf how good the
documentation is} -- for this project and for the finite element method in general.}  There are
extensive examples and online lectures as well as an active user group; all of which can
be accessed through the project's homepage.

\texttt{deal.II} is released under a \red{GNUGPL license???} so it is perfect for using in
batch simulations.  Compiling the \texttt{deal.II} libraries was described in
sec~\ref{subsec:SetupLinux} and on \texttt{deal.II}'s website.  Preliminary testing
suggests that the solutions from \texttt{deal.II} are better than for FlexPDE but this
is probably due to the convergence criteria of the solution for \texttt{deal.II} being
more suitable for QuMESHS.

As the website describes, \texttt{deal.II} is designed for running in Linux and QuMESHS's
compatibility with \texttt{deal.II} has only been tested in Linux.  The main disadvantage is
that \texttt{deal.II} is a C++ library and, in order to edit the device geometry, it is
recommended that you know what you are doing.  The geometries \red{which should be} described
in sec~\ref{sec:Examples} provide some default geometries with custom grids which can be
adapted but in order to implement different geometries, a good understanding of
\texttt{deal.II} is required.  \red{I would also be open to collaborations with any groups
in implementing specific geometries.}  The rest of this section will assume that the
reader knows the basics of using \texttt{deal.II}.

Probably the most difficult element of \texttt{deal.II}'s implementation in QuMESHS is
grid generation.  Firstly, \texttt{deal.II} uses hexahedra (lines, rectangles or cuboids)
to define its grid and I have not found a good meshing application for this.  Secondly,
in order to retrieve the potential values on the lattice for which the density is calculated
-- and to interpolate the density on the refined grids -- it is important not to
interpolate the values from the initial grid on to the refined grids as this is
\emph{\color{red} very} computationally expensive.  Therefore, the examples use an
``inter-grid mapping'' between the initial grid on which the density data is defined
and the adaptively refined grids.

This means that the initial grids consist of the regular lattice on which the density
is defined and the edges of the simulation domain.  \red{In general this can be
challenging as lattice points between regions (for example, of different materials)
must be moved to the same point.  See \texttt{CustomGridGenerator.cpp} for details.
This could be especially complicated for 3D structures.}

By defining the grid in this way, many cells are elongated.  It is suggested that
the refinement step should tag cells with a large aspect ratio for anisotropic
refinement.

Another issue that must be considered when calculating the initial potential is the
dopent distribution.  Unlike FlexPDE, the implementations of \texttt{deal.II} in
sec~\ref{sec:Examples} can interpolate the dopent density from 1D to 2- or 3-D.
This is done using linear interpolation between the data points.  As \texttt{deal.II}
refines the lattice adaptively, it is \emph{\color{red} very} important that the edges
of the dopent layers have a good resolution at the start of the calculation.  Otherwise,
after a given number of iterations, the grid refinement method will put lots of grid
points on one edge of the dopent layer.  This leads to a reduction of the total charge
in the dopent layer (see fig~\ref{fig:dealIIdopents}) which shifts the entire
potential.  The solution is to initially heavily refine the dopent layer so that the
interpolated density values are already well defined.  This leads to a better solution
being calculated in fewer iterations.  This is not relevant for calculating the
Newton step as the dopent density distribution is not included in $g(\phi)$.

The examples in sec~\ref{sec:Examples} define convergence of the \texttt{deal.II}
solution as the point where the potential on the lattice which defines the density
does not vary between refinements within a given tolerance (for example, $10^{-4}$V
in the initial potential calculations).  This seems to provide satisfactory
solutions.


\red{I will also apologise for the terrible layout and lack of commenting for the
\texttt{deal.II} implementation\ldots}


\subsection{Density Solvers}

The carrier density in a semiconductor heterostructure is calculated by solving
the many-body problem.  This is an impossible problem so a level of approximation
is needed to make the solution computationally tractable.  This section describes
the approximations available in QuMESHS and how they are implemented.  The theory
is described in sec~\ref{subsec:DensityTheory}.

\subsubsection{Thomas-Fermi Semiclassical Approximation --- 1D}



\subsubsection{Density Functional Theory --- 1D}

\subsubsection{Thomas-Fermi Semiclassical Approximation --- 2D}

\subsubsection{Effective-Band Approximation --- 2D}

\subsubsection{Density Functional Theory --- 2D}

\subsubsection{Spin-Orbit Interaction with Density Functional Theory --- 2D}

This is implemented but not tested\ldots

\subsubsection{Thomas-Fermi Semiclassical Approximation --- 3D}

This is implemented but not tested\ldots

\subsubsection{Transverse Thomas-Fermi Semiclassical Approximation --- 3D}

This is implemented but not tested\ldots

\subsubsection{Effective-Band Approximation --- 3D}

This is implemented but not tested\ldots

\subsubsection{Iterative Greens Function with Density Functional Theory --- 3D}

This is not implemented yet\ldots


\section{Data Structures and Designing Custom Density and Potential Solving Routines}

\section{Examples}
\label{sec:Examples}

\subsection{GaAs-AlGaAs Heterostructure}
\label{subsec:SimpleBandStructure}

The simplest practical problem which QuMESHS can solve is calculating the
band structure at a 285nm GaAs-AlGaAs interface for a modulation-doped wafer.
\red{This specific band structure is for W616 grown in the Semiconductor Physics
group at the University of Cambridge (can I say this and can I include the
growth sheet\ldots or a redacted version thereof)?}.  The growth specification
for a given wafer may provide a given doping density per layer.  However,
the actual density of available dopents is reduced due to \red{not sure why\ldots
check with Ian!}

QuMESHS will calculate the band structure and density both ``in the dark'',
fig~\ref{fig:BandStructureDark}, and ``after flashing'',
fig~\ref{fig:BandStructureFlash}.  The console output include the carrier
density calculated in the well after the quantum-mechanical density
functional calculation.  As noted above, flashing only increases the carrier
density by 30\% whilst experiment shows that it is increase by 50\%.  This
is because QuMESHS cannot calculate the contribution due to deep donors.
If you need to correct for this, increase the dopent density until the
correct carrier density is achieved.


\subsection{Double Quantum Well and Negative Capacitance}
\label{subsec:NegativeCapacitance}

This example aims to demonstrate QuMESHS's ability to run batch jobs, as well
as demonstrating possible issues with convergence and how this can be dealt with.

The example is based on the work of Eisenstein~\emph{et~al.}~\cite{Eisenstein:????}
and Milliard~\emph{et~al.}~\cite{Millard:1997} on negative capacitance of a
double well system under the application of a negative top-gate voltage.
\red{a bit more of a description of the problem and its physics\ldots}

The details for the batch run can be seen in \texttt{Input\_Parameters.txt}.
The top gate will be incremented by -10mV up to a maximum of -1V.  A batch job
can be run on a single computer by executing the code ``as is'' or it can be
run across a cluster by changing the \texttt{no\_runs} option to 1 and inputting
the batch number as the first console argument of QuMESHS \red{or possibly
\texttt{Solver\_Master.exe}\ldots}  If a batch job is run locally, the chemical
potential, dopent density and carrier density from the previous solution will
be used as the starting conditions for the next calculation.  This is only
true for 1D calculations or when the input option \texttt{hot\_start = true}.

The relevant output files is the carrier charge density for each voltage.  By
plotting the integrated carrier density across the two wells (note that you will
need to multiply the in-file values by \texttt{dz} in order to get the integrated
density and multiply the values by $10^{18}$ to convert from nm$^{-2}$ to
m$^{-2}$ as in the papers), with the density in the upper well on the $x$-axis
and the density in the lower well on the $y$-axis, the negative capacitance of
the lower well when the carrier density in the upper well is low can be seen,
fig~\ref{fig:NegativeCapacitance}.  A Matlab file for processing the data is
provided.

These results are at 1.5K and looking through the files generated by QuMESHS
it can be seen that the solutions at -0.32V and -0.80V have not converged.
These are the voltages where each of the wells depopulates.  At 1.5K, thermal
excitation into the wells when the chemical potential is level with the band
edge is very low so convergence requires very fine tuning of the solution at
such low densities.  Convergence at low temperatures and densities is a known
problem with density functional theory (see sec~\ref{subsec:DensityTheory}).
Possible solutions include increasing the temperature (to 4K, say) or reducing
the voltage step size (as this makes the previous solution a better ``first
guess'').  However, it is possible that the low density solution might not
provide much insight and an approximate, non-converged solution might be
adequate, as in this example.


\subsection{Split-Gate Heterostructure}

\subsection{Top-Gated Split Gates with Spontaneous Row Formation}
\label{subsec:TopAndSplitGate}
}


\appendix

\section{Material Properties}
\label{app:MaterialProperties}

\paragraph{GaAs}
\begin{itemize}
    \item{$E_g = 1424.0$ meV}
    \item{$\epsilon_r = 12.9$}
\end{itemize}

\paragraph{Al\textsubscript{x}Ga\textsubscript{1-x}As}
\begin{itemize}
    \item{$E_g = 1424.0 + 1247.0 x$ \ meV for $x < 0.45$}
    \item{$E_g = 1900.0 + 125.0 x + 143 x^2$ \ meV for $x \geq 0.45$}
    \item{$\epsilon_r = 10.1 x + 12.9 (1-x)$}
\end{itemize}

\paragraph{Air}
\begin{itemize}
    \item{$E_g = 0.5\times10^9$ meV}
    \item{$\epsilon_r = 1.0$}
\end{itemize}

\paragraph{PMMA}
\begin{itemize}
    \item{$E_g = 4400.0$ meV}
    \item{$\epsilon_r = 2.6$}
\end{itemize}

\paragraph{Metal}
\begin{itemize}
    \item{$E_g = 0.0$ meV}
    \item{$\epsilon_r = 1.0$}
\end{itemize}

\paragraph{In\textsubscript{x}Ga\textsubscript{1-x}As}
\begin{itemize}
    \item{$E_g = 1519.2 + 1583.7.0 x + 475.0 x^2$ \ meV}
    \item{$\epsilon_r = 14.6 x + 12.9 (1-x) + 13.5 x (1-x)$}
\end{itemize}

\paragraph{In\textsubscript{x}Al\textsubscript{1-x}As}
\begin{itemize}
    \item{$E_g = 2640.0 - 2280x$ \ meV}
    \item{$\epsilon_r = 14.6 x + 10.1 (1-x) + 12.5 x (1-x)$}
    \item Comments -- Alloy ratios of \texttt{x < 0.53} are not allowed.
\end{itemize}

\section{Geometry Definitions}
\label{app:GeometryDefinitions}

Under construction\ldots

\section{Known issues}
\begin{itemize}
\commentout{
    \item The \dealii{} solver for the Newton iteration step can stop without warning
    when solving the finite element matrix equation.  This error manifests itself when
    \dealii{} does not create a \texttt{x.dat}.   The exception (which comes from
    \dealii{}) says \texttt{Unknown exception!\textbackslash nAborting!}, followed by a stack
    trace.  Sometimes this can be solved by decreasing the accuracy of the matrix
    solver in \dealii{}.  This is done by changing the second argument of the \texttt{SolverControl}
    class in the \texttt{NewtonProblem<dim>::solve~()} method.  Another possible solution is
    to change the lattice spacing on which the density is calculated.
    %
}
    \item When the lattice spacing is too large, convergence is very difficult (if not impossible).
    This manifests itself with a very noisy potential and a density which randomly localises.
    The solution is to decrease the lattice spacing (if this is possible).
\end{itemize}

\end{document}
% ----------------------------------------------------------------
